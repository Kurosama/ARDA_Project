\chapter{Concluzioni}
In questa tesi si \`e voluto principalmente confrontare lo sviluppo dell'algoritmo
ARDA con 2 linguaggi di programmazione concettualmente diversi tra loro: Java e Python.
In seguito si \`e voluto apportare dei miglioramenti:
\begin{enumerate}
    \item Si \`e voluto riorganizzare il codice per una maggiore fruibilit\`a e leggibilit\`a
    \item Migliorare il calcolo della distanza tra punti e calcolo della matrice $M_{acc}$
\end{enumerate}
Come ultima parte, si \`e voluto implementare un'interfaccia grafica per poter visualizzare
i risultati in modo semplice e veloce, senza dimenticare la possibilit\`a di esportare i
dati in file CSV

\section{Considerazioni}
La conversione del codice da Python a Java, dopo l'attenta analisi del codice e di tutte
le sue funzionalit\`a e varianti, \`e stata abbastanza immediata, non trovando procedure o
funzioni differenti tra i due linguaggi.\\
Invece lo sviluppo dell'interfaccia grafica ha richiesto molta pi\`u attenzione.
Java non da nativamente le possibilit\`a di sviluppare interfacce tramite dei tool esterni
senza generare codice pesante e alle volte molto complesso a fronte delle semplicit\`a dell'interfaccia.
Lo stesso sistema di gestione dei componenti e della loro dimensione e posizione non \`e
intuitivo. Confrontandolo con linguaggi simili come C++ e lo stesso Python che, mediante
librerie grafiche specifiche danno la possibilit\`a creare interfacce grafiche, Java ha una
gestione abbastanza macchinosa per quando riguarda la composizione delle interfacce grafiche.
Stesso discorso vale per la mancanza di vere e proprie librerie per la gestione dei grafici
senza doversi appoggiare a libreria esterne a quelle di Java stessa.\\
Queste mancanze non hanno comportato criticit\`a nello sviluppo del programma ma non hanno
nemmeno agevolato lo sviluppo della parte grafica.

\section{Sviluppi futuri}
