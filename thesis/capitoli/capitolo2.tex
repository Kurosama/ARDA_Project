\chapter{Conversione e sviluppo}

\section{Analisi e scelte implementative}
Java e Python sono due linguaggi di programmazione simili ma diversi tra loro nati e progettati per necessit\`a differenti.\
Java nato agli inizi degli anni '90, \`e uno dei linguaggi di programmazione pi\`u conosciuti ed apprezzati
nel decennio passato. Ha avuto un notevole successo grazie all'esposione di internet e la possibilit\`a
di creare applicazioni client-server indipendenti dalla piattaforma. Simile al C++ come sintassi di base
ma senza alcune sue caratteristiche che potevano creare delle criticit\`a a livello di progettazione
(come l'aritmetica dei puntatori, ecc)\cite{new_2}, mentre le caratteristiche object oriented sia
a C++ ma soprattutto all'Objective C.\\
Il codice Java deve essere prima compilato per ottenere il cosiddetto \textit{bytecode} che poi verr\`a
eseguito attraverso una fase di interpretazione ad opera della Java Virtual Machine. Proprio questo procedimento
riesce a slegare i programmi dall'ambiente di compilazione ed esecuzione (da cui il motto \textit{Write once, Run everywhere}). \\
\\
Python, nato sempre agli inizi degli anni '90, \`e un un linguaggio multi-paradigma, che fa della dinamicit\`a,
semplicit\`a e flessibilit\`a i suoi principali obiettivi. Supporta la programmazione Object oriented e molte
caratteristiche di programmazione funzionale e di riflessione. Una sua caratteristica che lo contraddistingue \`e la
non tipizzazione delle variabili.

\subsection{Conversione}
Sono proprio le differenze appena descritte che si sono contraddistinte durante la conversione del progetto da Python a Java.
La struttura pi\`u rigida di Java ha richiesto un'analisi delle variabili, delle funzioni e degli oggetti utilizzati che,
a volte, ha richiesto una conversione della tipologia delle variabili oppure una divisione per ogni tipologia utilizzata dal programma.\\
Sebbene questa rigidit\`a abbia richiesto un'analisi preventiva del codice, in seguito ha reso lo sviluppo pi\`u agevole in quanto
il flusso dei dati rimane coerente visto che non pu\`o variare di tipologia.

\subsection{Unificazione dei valori di default}
Nel progetto originale si era preferito dividere sia l'esecuzione dei test tra il percorso di Udine con quello ad Atlanta,
sia il calcolo delle matrici di importanza con l'algoritmo SpaceRank dagli altri metodi, questo perch\'e si \`e preferito
concentrare lo sviluppo degli algoritmi e dei test per i casi specifici.
Tale approccio ha portato alla realizzazione di quattro distinti flussi di codice simili tra loro ma con variabili e funzioni
indipendenti e non riutilizzabili.\\
Durante l'analisi iniziale, necessaria per definire funzioni e variabili, si \`e preferito unificare le procedure simili in modo da
semplificare e diminuire il codice necessario per il programma. Anche le strutture dati sono risultate simili optando per
unificare la gestione del calcolo della matrice di importanza.\\

\subsection{Modifiche fatte}
Durante lo sviluppo dell'applicativo si sono voluti apportare dei cambiamenti
a livello di calcolo.
\subsubsection{Calcolo della distanza}
Nella versione precedente, per calcolare le distanze tra due punti successivi,
ottenuti dalle rilevazioni, veniva usato il calcolo della distanza lineare tra 2 punti. Nel nostro caso
non possiamo considerare i nostri spostamenti su un piano ma spostamenti su una
superficie sferica. Per questo si \`e preferito sostituire
\begin{equation}
d =  \sqrt{(x_{1}-x_{2})^2 + (y_{1}-y_{2})^2}
\end{equation}
con il calcolo della distanza su una superficie sferica
\begin{equation}
d = r_{earth} * \arccos{( (\sin{lat_{1}} * \sin{lat_{2}}) + (\cos{lat_{1}} * \cos{lat_{2}} * \cos{(lat_{1} - lat_{2})} ) )}
\end{equation}
Questa variazione rende la distanza calcolata pi\`u vicina alla realt\`a, sopratutto
quando i due punti sono molto lontani tra loro, in alcuni casi limite, pu\`o ancora non essere precisa.\\
Negli spostamenti presi in esame questo cambiamento porta un miglioramento marginale ma si \`e preferito
introdurla sopratutto per avere una base di sviluppo pi\`u accurata.

\subsubsection{Calcolo della matrice accelerazione}
Nel calcolo delle matrici ARDA, per ottenere la matrice delle accelerazione $M_{acc}$ si esegue, sulla
matrice $M_{imp}$, una funzione gradiente per calcolare i valori del vettore in tutte le zone.
La matrice generata ha ottimi risultati con punti di importanza distanti tra loro
ma con punti vicini il calcolo del vettore ha qualche incongruenza nelle zone adiacenti.\\
Utilizzando l'algoritmo di Sobel, questo distribuisce con maggior equit\`a le forze anche in presenza di
molti punti di interessi vicini tra loro.\\
Sia nel precedente progetto che in questo, la matrice di accelerazione viene calcolata dopo un solo processo
della matrice $M_{imp}$ calcolando le forze solo nelle zone adiacenti a quelle con dei valori di interesse.
Pensando proprio all'idea su cui si basa l'utilizzo delle matrici di accelerazione si pu\`o capire quanto possa
essere limitativo eseguire questo calcolo solo in queste zone. Questo pu\`o portare ad un ulteriore sviluppo e
confronto con il calcolo della matrice di accelerazione con pi\`u ricalcoli sulla matrice risultante
\begin{equation}
M_{acc} = G^{n}(M_{imp})
\end{equation}

\section{Interfaccia Java}
Un'ulteriore obbiettivo del progetto \`e creare un'interfaccia user-frienly per poter eseguire i setup e visualizzare
i risultati in modo semplice e veloce. Nel progetto precedente bisognava eseguire i setup dei file,
directory e impostazioni direttamente nei vari file del codice.\\

\begin{figure}[h]
    \begin{center}
    \includegraphics[scale=0.5]{Interface.png}
    \caption[IntJava]{Interfaccia Java}
    \label{etichetta}
    \end{center}
\end{figure}

Adesso l'utente ha la possibilit\`a di impostare:
\begin{enumerate}
    \item i file mediante una semplice selezione tramite un finder navigando nel filesytem
    \item selezionare le dimensione delle celle e gli indici utilizzati per il calcolo
    \item visualizzare un plot della matrice di importanza.
\end{enumerate}
Per l'esecuzione dei test si \`e scelto di dividere il calcolo dal salvataggio dei risultati mediante i file CSV
gi\`a implementati nella versione precedente, con la possibilit\`a di visualizzare solamente il dato preso in esame.
Proprio per rendere facilmente utilizzabile la funzione di visualizzazione dei dati, la sezione dei grafici \`e
divisa in varie schede (come si pu\`o vedere dall'immagine) generate in base alle opzioni selezionate.

\subsection{Visualizzazione plot matrici}
Questa funzionalit\`a ha per\`o tolto la possibilit\`a di selezione multipla della grandezza delle zone di importanza
e impostato un limite di 2 possibili tipi di indice.

\begin{figure}[h]
    \begin{center}
    \includegraphics[scale=0.5]{Interface_plot.png}
    \caption[IntJava]{Interfaccia Java}
    \label{etichetta}
    \end{center}
\end{figure}

Infatti la visualizzazione e la gestione delle schede sarebbe stata troppo dispersiva data la possibilit\`a di
generare 25 casi separati con almeno 7 grafici differenti per ognuno di essi ed un totale di 175 schede massime.
Con le limitazioni imposte si arriva a un massimo di 14 schede divise in 2 righe, sufficienti per avere subito
dei grafici e poter fare un confronto veloce tra 2 tipi di indice differenti.\\
Questa piccola variazione per\`o non influenza la generazione dei file CSV con i risultati. Essi non vengono cancellati
e, con il lancio di tutte le procedure mediante l'interfaccia, possono essere facilmente reperibili per effettuare
analisi direttamente dai dati che normalmente si visualizzano nei grafici.
