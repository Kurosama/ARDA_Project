\chapter{Conversione e sviluppo}

\section{Analisi e scelte implementative}
\subsection{Conversione}
- Python molto libero nella definizione di variabili e funzione (creazione e richiamo)\\
- Java molto rigido nella definizione\\
- Gestione dei dati simile ma differente\\

\subsection{Unificazione dei valori di default}
- Prima erano 2-3 versione dei dati (conv metri lat-long)\\
- Unificazione delle procedure di test sia con Mimp che SR\\


%----------------------------
\subsection{Modifiche fatte}
Durante lo sviluppo dell'applicativo si sono voluti apportare dei cambiamenti
a livello di calcolo.
\subsubsection{Calcolo distanza tra 2 punti}
Nella versione precedente, per calcolare le distanze tra due punti successivi,
ottenuti dalle rilevazioni, veniva usato il calcolo tra 2 punti. Nel nostro caso
non possiamo considerare i nostri spostamenti su un piano ma spostamenti su una
superficie sferica. Per questo si è preferito sostituire
\begin{equation}
d =    \dots
\end{equation}
con il calcolo della distanza in una superficie sferica
\begin{equation}
d =
\end{equation}
Questa variazione rende il distanza calcolata più visina alla realta, sopratutto
quando la distanza tra due punti aumenta.\\
Nei casi presi in esame non porta a miglioramenti significati ma si e preferito
introdurla per avere una base di sviluppo piu accurata.
%----------------------------------------

\subsubsection{Metodo del calcolo dell'autovettore}


\subsubsection{calcolo della matrice delle addiacenze}
 (con Sobel)

\section{Interfaccia Java}
- descrizione dell'interfaccia con varie immagini (da decidere se mettere in appendice)\\
\subsection{Visualizzazione plot matrici}
- Mostrare graficamente le matrici di importanza (verificare il plot 2d e 3d)\\
- Mostrare graficamente i risultati ottenuti (gestire diversamente la selezione dei parametri)\\
!! Controllare che le API Google funzionino ancora oppure vedere per OpenStreetView\\
\subsection{Esportazione risultati}
- Breve spiegazione del perche scegliere i file CSV\\
!! Verificare la possibilita di generare file con grafici in xls o xml\\
