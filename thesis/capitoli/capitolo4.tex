\chapter{Concluzioni}
In questa tesi si $\grave{e}$ voluto principalmente confrontare lo sviluppo dell'algoritmo
ARDA con 2 linguaggi di programmazione concettualmente diversi tra loro: Java e Python.
In seguito si $\grave{e}$ voluto apportare dei miglioramenti:
\begin{enumerate}
    \item Si $\grave{e}$ voluto riorganizzare il codice per una maggiore fruibilit$\grave{a}$ e leggibilit$\grave{a}$
    \item Migliorare il calcolo della distanza tra punti e calcolo della matrice $M_{acc}$
\end{enumerate}
Come ultima parte, si $\grave{e}$ voluto implementare un'interfaccia grafica per poter visualizzare
i risultati in modo semplice e veloce, senza dimenticare la possibilit$\grave{a}$ di esportare i
dati in file CSV

\section{Considerazioni}
La conversione del codice da Python a Java, dopo l'attenta analisi del codice e di tutte
le sue funzionalit$\grave{a}$ e varianti, $\grave{e}$ stata abbastanza immediata, non trovando procedure o
funzioni differenti tra i due linguaggi.\\
Invece lo sviluppo dell'interfaccia grafica ha richiesto molta pi$\grave{u}$ attenzione.
Java non da nativamente le possibilit$\grave{a}$ di sviluppare interfacce tramite dei tool esterni
senza generare codice pesante e alle volte molto complesso a fronte delle semplicit$\grave{a}$ dell'interfaccia.
Lo stesso sistema di gestione dei componenti e della loro dimensione e posizione non $\grave{e}$
intuitivo. Confrontandolo con linguaggi simili come C++ e lo stesso Python che, mediante
librerie grafiche specifiche danno la possibilit$\grave{a}$ creare interfacce grafiche, Java ha una
gestione abbastanza macchinosa per quando riguarda la composizione delle interfacce grafiche.
Stesso discorso vale per la mancanza di vere e proprie librerie per la gestione dei grafici
senza doversi appoggiare a libreria esterne a quelle di Java stessa.\\
Queste mancanze non hanno comportato criticit$\grave{a}$ nello sviluppo del programma ma non hanno
nemmeno agevolato lo sviluppo della parte grafica.

\section{Sviluppi futuri}
