\chapter{Concluzioni}
In questa tesi si e voluto principamente confrontare lo sviluppo dell'algoritmo
ARDA con 2 linguaggi di programmazione concettualmente diversi tra loro: Java e Python.
In seguito si e voluto apportare delle migliorimenti:
\begin{enumerate}
    \item Si e voluto ri-organizzare il codice per una maggiore fruibilita e leggibilita
    \item Migliorare il calcolo della distanza tra punti e calcolo della matrice $M_{acc}$
\end{enumerate}
Come ultima parte, si e voluto implementare un'interfaccia grafica per poter visualiizare
i risultati in modo semplice e veloce, senza dimenticare la possibilita di esportare i
dati in file CSV

\section{Considerazioni}
La conversione del codice da Python a Java, dopo l'attenta analisi del codice e di tutte
le sue funzionalita e varianti, e stata abbastanza immedita, non trovando procedure o
funzioni differenti tra i due linguaggi.\\
Invece lo sviluppo dell'interfaccia grafica ha richiesto molta piu attenzione.
Java non da nativamente le possibilita di sviluppare interfacce tramite dei tool esperni
senza generare codice pesante e alle volte molto complesso a fronte delle semplicita dell'interfaccia.
Lo stesso sistema di gestione dei componenti e della loro dimensione e posione non e
intuitivo. Confrontandolo con linguaggi simili come C++ e lo stesso Python che, mediante
librerie grafiche specifiche danno la possibilita creare interfacce grafiche, Java ha una
gestione abbastanza macchinosa per quando riguarda la composizione delle interfacce grafiche.
Stesso discorso vale per la mancanza di vere e proprie librerie per la gestione dei grafici
senza doversi appoggiare a libreria esterne a quelle di Java stessa.\\
Queste mancanze non hanno comportato criticita nello sviluppo del programma ma non hanno
nemmeno aggevolato lo sviluppo della parte grafica.

\section{Sviluppi futuri}
