\chapter{Comparative prestazionali}
Questo capitolo \`e stato inserito pensando a come questi algoritmi se possono
essere utilizzati su dispositivi usati tutti i giorni oppure necessitano di
ulteriori miglioramenti e/o determinate macchine o ambianti.\\
Non avendo un insieme di risultati abbastanza grande da fare delle statistiche
che possano descrivono efficacemente i vari casi, questo capitolo \`e puramente
teorico e cerca di dare un'ulteriore ambito di sviluppo per le ricerche future.\\
\\
Tutti i risultati che si utilizzano fanno riferimento all'esecuzione non solo
dell'algoritmo di previsione, ma anche di lettura dello storico e dei persorsi.
Questo perch\`e, visto la scelta di eseguire il codice in un sistema con risorse
limitate (RaspBerry Pi) si vuole tener conto anche della velocità di accesso alla memoria.
L'algoritmo \`e stato eseguito 120 volte per tutti gli indici, tranne per lo
\textit{spaceRank} che \`e stato eseguito solo 6 volte. Queste esecusioni sono ripartite
equamente per 3 differenti dimensioni delle localit\`a (50,100,500 metri).
\\
Le tre macchine che si utilizzeranno per i vari confronti avranno le seguenti configurazioni:
\begin{itemize}
\item [M1] Processore Intel i7 \cite{new_3} e 8Gb di RAM DDR3
\item [M2] Processore Intel Core 2 Duo \cite{new_4} e 4Gb di RAM DDR2
\item [M3] Processore ARM \cite{new_5} con 512Mb di SDRAM
\end{itemize}

\section{Confronto tra Java e Python}
Per il confronto tra le due versioni dell'algoritmo si \`e scelto di utilizzare la
configurazione M1. I risultati visibili in tabella sono la media dei vari tempi
rilevati nell'area di Atlanta.
\begin{table}[]
\centering
\caption{Tempi di esecuzione algoritomo ARDA (M1)}
\label{bcmrk_ARDA}
\begin{tabular}{lllll}
\hline
\multicolumn{1}{|c|}{}                        & \multicolumn{1}{|c|}{Java}    & \multicolumn{1}{|c|}{Python} \\
\multicolumn{1}{|c|}{$\sharp$\textit{visits}} & \multicolumn{1}{|c|}{1,543 s} & \multicolumn{1}{|c|}{1,613 s}\\
\multicolumn{1}{|c|}{\textit{avgTime}}        & \multicolumn{1}{|c|}{0,979 s} & \multicolumn{1}{|c|}{0,975 s}\\
\multicolumn{1}{|c|}{\textit{totTime}}        & \multicolumn{1}{|c|}{0,976 s} & \multicolumn{1}{|c|}{0,978 s}\\
\multicolumn{1}{|c|}{\textit{ComLin}}         & \multicolumn{1}{|c|}{0,987 s} & \multicolumn{1}{|c|}{0,985 s}\\
\multicolumn{1}{|c|}{\textit{spaceRank}}     & \multicolumn{1}{|c|}{ 67 h}    & \multicolumn{1}{|c|}{67 h} \\
\hline
\end{tabular}
\end{table}
La cosa che subito colpisce sono i valori di calcolo dello \textit{spaceRank}.
I quasi 3 giorni necessari al calcolo sono dovuti principalmente al calcolo dell'autovettore
della matrice $M_{imp}$ che impiega il 90\% del tempo di esecuzione. Infatti il
calcolo delle statistiche \`e uguale per tutti visto che valuta allo stesso modo le matrici
di importanza, accelerazione e potenze allo stesso modo, indipendentemente dall'indice
usato per calcolarle.\\
Valutando i risultati rimanenti vi sono dei leggeri miglioramenti e questi fa ben sperate
per l'esecuzione con insiemi di dati pi\`u corposi.

\section{Confronto temporate}
In questa sezione verr\`a fatta una comparativa simile alla precedente ma esaminando
la differenza tra due macchine di generazioni differenti per capire quanto la tecnologia
sia migliorata, tenendo conto che prestazioni simili alla configurazione M2 esame stanno
per essere raggiunte da smartphone e tablet comunemente utilizzati dall'utente medio.
\begin{table}[]
\centering
\caption{Tempi di esecuzione algoritomo ARDA (M2)}
\label{bcmrk_ARDA}
\begin{tabular}{lllll}
\hline
\multicolumn{1}{|c|}{}                        & \multicolumn{1}{|c|}{Java} & \multicolumn{1}{|c|}{Python}  & \multicolumn{1}{|c|}{Diff. da M1} \\
\multicolumn{1}{|c|}{$\sharp$\textit{visits}} & \multicolumn{1}{|c|}{ s}   & \multicolumn{1}{|c|}{ s} & \multicolumn{1}{|c|}{ s}             \\
\multicolumn{1}{|c|}{\textit{avgTime}}        & \multicolumn{1}{|c|}{ s}   & \multicolumn{1}{|c|}{ s} & \multicolumn{1}{|c|}{ s}             \\
\multicolumn{1}{|c|}{\textit{totTime}}        & \multicolumn{1}{|c|}{ s}   & \multicolumn{1}{|c|}{ s} & \multicolumn{1}{|c|}{ s}             \\
\multicolumn{1}{|c|}{\textit{ComLin}}         & \multicolumn{1}{|c|}{ s}   & \multicolumn{1}{|c|}{ s} & \multicolumn{1}{|c|}{ s}             \\
\multicolumn{1}{|c|}{\textit{spaceRank}}      & \multicolumn{1}{|c|}{ h}   & \multicolumn{1}{|c|}{ h}    & \multicolumn{1}{|c|}{ s}\\
\hline
\end{tabular}
\end{table}
Anche qui i risultati sono incoraggianti visto il miglioramento nell'ordine del
\% sui tempi di elaborazione. Questo fa ben sperare visto che la tecnoligia mobile
migliora di anno in anno con processori sempre pi\`u performanti e vicini ai PC.
