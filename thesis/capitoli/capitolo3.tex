\chapter{Comparative sui risultati ottenuti}

\section{Matrici d'importanza}
\subsection{Dati Python}
\subsection{Dati Java}

\section{ARDA}
\subsection{Dati Python}
\subsection{Dati Java}

\section{Valutazioni complessive}
Come gia detto nell'introduzione e nel capitolo 2, le varie modifiche apportate
al codice hanno portato differenze marginali al livello di risultati complessivi.
Il limitato miglioramente e dovuto alla dimensione delle aree prese in esame, troppo
piccole per poter vedere i miglioramente ottenibili delle nuove funzioni introdotte.\\
Questo pero non va incriminato alla scelta dei dati ma proprio alla natura dei
dati da analizzare, che cerca di prevedere le distanazioni su una serie di spostamenti
ottenuti monitorando delgi utenti nelle loro spostamente abbituali. Ed e proprio la
natura ripetitiva dei spostamenti a rendere piccola la zona di interesse, visto che
difficilemente una persona compie spostamente giornalieri su un'area di centinaia di
chilometri quadrati.
