\begin{thebibliography}{999}
\bibitem{cit_48} A.Dey, \textit{Understanding and using context}, Personal and Ubiquitous Computing, vol. 5, no. 1, pp 4-7, 2001
\bibitem{cit_47} Chaoming Song, Zehui Qu, Nicholas Blumm, Albert-Laszlo Barabasi. \textit{Limits of Predictability in Human Mobility}, Science 19 February 2010: Vol.327 . no 5968, pp.1018-1021
\bibitem{new_1} Luca Snaidero, \textit{Valutazione sperimentale dell'algoritmo ARDA per la previsione di traiettorie}, 2009
\bibitem{cit_42} J. Krumm and E. Horvitz. \textit{Predestination: Inferring destinations from partial trajectories ubicomp.} In Springer, editor, Lecture Notes in Computer Science, volume 4206, pages 243-260, 2006.
\bibitem{cit_43} D. Ambrosin and A. Sciomachen. \textit{A gravitational approach for locating new services in urban areas}. In Proceedings of the 16th Mini - EURO Conference and 10th Meeting of EWGT, pages 194-199, Poznan, Poland, 2005. Poznan University of Technology.
\bibitem{cit_44} M. C. Gonzalez, C. A. Hidalgo, and A. L. Barabasi. \textit{Understanding individual human mobility patterns}. Nature, 435(7196):779-782, 2008.
\bibitem{cit_45} L. Page, S. Brin, R. Motwani, and T. Winograd. \textit{The pagerank citation ranking: Bringing order to the web.} Technical report, Stanford Digital Library Technologies Project, 1998.
\bibitem{cit_49} S. De Sabbata, \textit{Pre-destinazione: modelli ed esperimenti per la previsione di traiettorie}, Tesi laurea specialistica Universita degli Studi di Udine.
\end{thebibliography}
